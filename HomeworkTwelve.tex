% Options for packages loaded elsewhere
\PassOptionsToPackage{unicode}{hyperref}
\PassOptionsToPackage{hyphens}{url}
%
\documentclass[
]{article}
\usepackage{lmodern}
\usepackage{amssymb,amsmath}
\usepackage{ifxetex,ifluatex}
\ifnum 0\ifxetex 1\fi\ifluatex 1\fi=0 % if pdftex
  \usepackage[T1]{fontenc}
  \usepackage[utf8]{inputenc}
  \usepackage{textcomp} % provide euro and other symbols
\else % if luatex or xetex
  \usepackage{unicode-math}
  \defaultfontfeatures{Scale=MatchLowercase}
  \defaultfontfeatures[\rmfamily]{Ligatures=TeX,Scale=1}
\fi
% Use upquote if available, for straight quotes in verbatim environments
\IfFileExists{upquote.sty}{\usepackage{upquote}}{}
\IfFileExists{microtype.sty}{% use microtype if available
  \usepackage[]{microtype}
  \UseMicrotypeSet[protrusion]{basicmath} % disable protrusion for tt fonts
}{}
\makeatletter
\@ifundefined{KOMAClassName}{% if non-KOMA class
  \IfFileExists{parskip.sty}{%
    \usepackage{parskip}
  }{% else
    \setlength{\parindent}{0pt}
    \setlength{\parskip}{6pt plus 2pt minus 1pt}}
}{% if KOMA class
  \KOMAoptions{parskip=half}}
\makeatother
\usepackage{xcolor}
\IfFileExists{xurl.sty}{\usepackage{xurl}}{} % add URL line breaks if available
\IfFileExists{bookmark.sty}{\usepackage{bookmark}}{\usepackage{hyperref}}
\hypersetup{
  pdftitle={Homework 12},
  pdfauthor={Alison Hall},
  hidelinks,
  pdfcreator={LaTeX via pandoc}}
\urlstyle{same} % disable monospaced font for URLs
\usepackage[margin=1in]{geometry}
\usepackage{color}
\usepackage{fancyvrb}
\newcommand{\VerbBar}{|}
\newcommand{\VERB}{\Verb[commandchars=\\\{\}]}
\DefineVerbatimEnvironment{Highlighting}{Verbatim}{commandchars=\\\{\}}
% Add ',fontsize=\small' for more characters per line
\usepackage{framed}
\definecolor{shadecolor}{RGB}{248,248,248}
\newenvironment{Shaded}{\begin{snugshade}}{\end{snugshade}}
\newcommand{\AlertTok}[1]{\textcolor[rgb]{0.94,0.16,0.16}{#1}}
\newcommand{\AnnotationTok}[1]{\textcolor[rgb]{0.56,0.35,0.01}{\textbf{\textit{#1}}}}
\newcommand{\AttributeTok}[1]{\textcolor[rgb]{0.77,0.63,0.00}{#1}}
\newcommand{\BaseNTok}[1]{\textcolor[rgb]{0.00,0.00,0.81}{#1}}
\newcommand{\BuiltInTok}[1]{#1}
\newcommand{\CharTok}[1]{\textcolor[rgb]{0.31,0.60,0.02}{#1}}
\newcommand{\CommentTok}[1]{\textcolor[rgb]{0.56,0.35,0.01}{\textit{#1}}}
\newcommand{\CommentVarTok}[1]{\textcolor[rgb]{0.56,0.35,0.01}{\textbf{\textit{#1}}}}
\newcommand{\ConstantTok}[1]{\textcolor[rgb]{0.00,0.00,0.00}{#1}}
\newcommand{\ControlFlowTok}[1]{\textcolor[rgb]{0.13,0.29,0.53}{\textbf{#1}}}
\newcommand{\DataTypeTok}[1]{\textcolor[rgb]{0.13,0.29,0.53}{#1}}
\newcommand{\DecValTok}[1]{\textcolor[rgb]{0.00,0.00,0.81}{#1}}
\newcommand{\DocumentationTok}[1]{\textcolor[rgb]{0.56,0.35,0.01}{\textbf{\textit{#1}}}}
\newcommand{\ErrorTok}[1]{\textcolor[rgb]{0.64,0.00,0.00}{\textbf{#1}}}
\newcommand{\ExtensionTok}[1]{#1}
\newcommand{\FloatTok}[1]{\textcolor[rgb]{0.00,0.00,0.81}{#1}}
\newcommand{\FunctionTok}[1]{\textcolor[rgb]{0.00,0.00,0.00}{#1}}
\newcommand{\ImportTok}[1]{#1}
\newcommand{\InformationTok}[1]{\textcolor[rgb]{0.56,0.35,0.01}{\textbf{\textit{#1}}}}
\newcommand{\KeywordTok}[1]{\textcolor[rgb]{0.13,0.29,0.53}{\textbf{#1}}}
\newcommand{\NormalTok}[1]{#1}
\newcommand{\OperatorTok}[1]{\textcolor[rgb]{0.81,0.36,0.00}{\textbf{#1}}}
\newcommand{\OtherTok}[1]{\textcolor[rgb]{0.56,0.35,0.01}{#1}}
\newcommand{\PreprocessorTok}[1]{\textcolor[rgb]{0.56,0.35,0.01}{\textit{#1}}}
\newcommand{\RegionMarkerTok}[1]{#1}
\newcommand{\SpecialCharTok}[1]{\textcolor[rgb]{0.00,0.00,0.00}{#1}}
\newcommand{\SpecialStringTok}[1]{\textcolor[rgb]{0.31,0.60,0.02}{#1}}
\newcommand{\StringTok}[1]{\textcolor[rgb]{0.31,0.60,0.02}{#1}}
\newcommand{\VariableTok}[1]{\textcolor[rgb]{0.00,0.00,0.00}{#1}}
\newcommand{\VerbatimStringTok}[1]{\textcolor[rgb]{0.31,0.60,0.02}{#1}}
\newcommand{\WarningTok}[1]{\textcolor[rgb]{0.56,0.35,0.01}{\textbf{\textit{#1}}}}
\usepackage{graphicx,grffile}
\makeatletter
\def\maxwidth{\ifdim\Gin@nat@width>\linewidth\linewidth\else\Gin@nat@width\fi}
\def\maxheight{\ifdim\Gin@nat@height>\textheight\textheight\else\Gin@nat@height\fi}
\makeatother
% Scale images if necessary, so that they will not overflow the page
% margins by default, and it is still possible to overwrite the defaults
% using explicit options in \includegraphics[width, height, ...]{}
\setkeys{Gin}{width=\maxwidth,height=\maxheight,keepaspectratio}
% Set default figure placement to htbp
\makeatletter
\def\fps@figure{htbp}
\makeatother
\setlength{\emergencystretch}{3em} % prevent overfull lines
\providecommand{\tightlist}{%
  \setlength{\itemsep}{0pt}\setlength{\parskip}{0pt}}
\setcounter{secnumdepth}{-\maxdimen} % remove section numbering

\title{Homework 12}
\author{Alison Hall}
\date{4/27/2020}

\begin{document}
\maketitle

\begin{enumerate}
\def\labelenumi{\arabic{enumi}.}
\tightlist
\item
  For this exerise, use your newly-developed ggplot chops to create some
  nice graphs from your own data (If you do not have a good data frame
  to use for graphics, use one of the many built-in data frames from R
  (other than mpg, which we are using in class)). Experiment with
  different themes, theme base sizes, aesthetics, mappings, and
  faceting. When you are finished, try exporting them to high quality
  pdfs, jpgs, eps files, or other formats that you would use for
  submission to a journal. In this exercise, I encourage you to improve
  your graphics with elements that we have not (yet) covered in ggplot.
  For example, can you change the labels on a facet plot so that they
  are more informative than the variable names that are supplied from
  your data frame? Can you figure out how to add text annotations, lines
  and arrows to your graph? Can you figure out how to use custom colors
  that you have chosen for your fills and lines? Your resources for
  these explorations are google, Stack Overflow -- and Lauren!
\end{enumerate}

Step one: load libraries and data set. Data set used here lists
different treatment types- steady and gradual and how long it took an
urchin to right itself. There were 4 individuals in the steady group
(Steady3,Steady4, Steady7, Steady8) and 3 individuals in the gradual
treatment group (Gradual1, Gradual2, Gradual6).

\begin{Shaded}
\begin{Highlighting}[]
\KeywordTok{pdf}\NormalTok{(}\DataTypeTok{file =} \StringTok{"/Users/Alison's Laptop/Documents/UVM/Computational Biology/HallBio381/MyPlots.pdf"}\NormalTok{, }\DataTypeTok{width=}\DecValTok{4}\NormalTok{, }\DataTypeTok{height=}\DecValTok{4}\NormalTok{)}
\end{Highlighting}
\end{Shaded}

Make plots! here's a boxplot

\begin{Shaded}
\begin{Highlighting}[]
\KeywordTok{pdf}\NormalTok{(}\DataTypeTok{file =} \StringTok{"/Users/Alison's Laptop/Documents/UVM/Computational Biology/HallBio381/MyPlots.pdf"}\NormalTok{, }\DataTypeTok{width=}\DecValTok{4}\NormalTok{, }\DataTypeTok{height=}\DecValTok{4}\NormalTok{)}
\NormalTok{BoxPlot <-}\StringTok{ }\KeywordTok{ggplot}\NormalTok{(}\DataTypeTok{data=}\NormalTok{SalinityData3, }\KeywordTok{aes}\NormalTok{(}\DataTypeTok{x=}\NormalTok{Individual, }\DataTypeTok{y=}\NormalTok{RightingTime_seconds)) }\OperatorTok{+}\StringTok{ }\KeywordTok{geom_boxplot}\NormalTok{(}\KeywordTok{aes}\NormalTok{(}\DataTypeTok{fill=}\NormalTok{Treatment))}
\KeywordTok{print}\NormalTok{(BoxPlot)}
\KeywordTok{dev.off}\NormalTok{()}
\end{Highlighting}
\end{Shaded}

\begin{verbatim}
## pdf 
##   2
\end{verbatim}

This version of the box plots shows the distribution of a righting times
doe an individual urchin across the duration of the experiment. Each dot
is righting time on a different day of the study.

\begin{Shaded}
\begin{Highlighting}[]
\NormalTok{jitterboxplot <-}\StringTok{ }\KeywordTok{ggplot}\NormalTok{(}\DataTypeTok{data=}\NormalTok{SalinityData3, }\KeywordTok{aes}\NormalTok{(}\DataTypeTok{x=}\NormalTok{Individual, }\DataTypeTok{y=}\NormalTok{RightingTime_seconds, }\DataTypeTok{color=}\NormalTok{Treatment)) }\OperatorTok{+}
\StringTok{  }\KeywordTok{geom_boxplot}\NormalTok{() }\OperatorTok{+}\StringTok{ }\KeywordTok{geom_jitter}\NormalTok{(}\DataTypeTok{position =} \KeywordTok{position_jitter}\NormalTok{(}\FloatTok{0.2}\NormalTok{))}
\KeywordTok{print}\NormalTok{(jitterboxplot)}
\end{Highlighting}
\end{Shaded}

\includegraphics{HomeworkTwelve_files/figure-latex/unnamed-chunk-3-1.pdf}

Violin plots with box plots allow us to see the density of righting time
data as well. I like this one, so I saved it to a pdf!

\begin{Shaded}
\begin{Highlighting}[]
\NormalTok{violinbox <-}\StringTok{ }\KeywordTok{ggplot}\NormalTok{(SalinityData3) }\OperatorTok{+}\StringTok{ }\KeywordTok{geom_violin}\NormalTok{(}
  \KeywordTok{aes}\NormalTok{(}\DataTypeTok{x=}\NormalTok{Individual, }\DataTypeTok{y=}\NormalTok{RightingTime_seconds, }
    \DataTypeTok{fill=}\NormalTok{Treatment)) }\OperatorTok{+}\StringTok{ }\KeywordTok{geom_boxplot}\NormalTok{(}\KeywordTok{aes}\NormalTok{(}\DataTypeTok{x=}\NormalTok{Individual, }
                                        \DataTypeTok{y=}\NormalTok{RightingTime_seconds),}
                                        \DataTypeTok{width=}\FloatTok{0.3}\NormalTok{, }
                                        \DataTypeTok{alpha=}\FloatTok{0.4}\NormalTok{)}
\KeywordTok{ggsave}\NormalTok{(}\StringTok{"violinbox.pdf"}\NormalTok{) }\CommentTok{#saving this last one to a pdf :) }
\end{Highlighting}
\end{Shaded}

\begin{verbatim}
## Saving 6.5 x 4.5 in image
\end{verbatim}

\end{document}
